\subsection{The Siberian Ring Source of Photons (SKIF) and the Development of
its Digital Twin}

The Siberian Ring Source of Photons (SKIF) represents a significant advancement
in the field of synchrotron radiation sources. As a 4+ generation synchrotron
with six research workstations, SKIF generates synchrotron radiation with a
power density of 92 kW/mrad², making it a key research tool in physics and
scientific research \cite{Kabov2021Device, Grishina2020Analysis}. This 3-GeV
electron energy SR source, developed in Novosibirsk, is designed to provide a
new level of capabilities for scientific experiments, particularly in the
fields of materials science and physics \cite{Gurov2020Injection,
Maltseva2022Beam}.

The development of a digital twin for SKIF offers numerous benefits, enhancing
the facility's operational efficiency and research capabilities. A digital
twin, in this context, is a virtual representation that mirrors the physical
attributes and dynamics of SKIF, allowing for real-time monitoring, simulation,
and optimization of the facility's performance. This technology is particularly
beneficial in complying with the precise requirements for the mutual
positioning of electromagnetic axes of the elements during installation, which
is crucial for the optimal functioning of SKIF \cite{Polyansky2022The}.

One specific component of SKIF that is of particular interest is the diamond
window. This component is essential in the facility's operation, serving as a
robust and transparent medium for synchrotron radiation. The development of a
digital twin for the diamond window involves creating a detailed virtual model
that can simulate its behavior under various operational conditions. This model
can predict how the diamond window will react to different radiation
intensities, thermal loads, and mechanical stresses, thereby aiding in its
design, testing, and maintenance.

The digital twin of the diamond window allows for the efficient modeling and
testing of new control opportunities, improving the operation and longevity of
this critical component. It enables the simulation of temperature, stress, and
strain distribution within the diamond window, ensuring that it can withstand
the intense conditions of SKIF without degradation \cite{Kabov2021Device}.
Additionally, the digital twin can be used to study the impact of different
cooling strategies on the diamond window, optimizing its thermal management and
ensuring its performance remains consistent.

In summary, the development of a digital twin for the Siberian Ring Source of
Photons (SKIF), particularly for its diamond window, represents a significant
step forward in the digitalization of scientific research facilities. This
technology not only enhances the operational efficiency and research
capabilities of SKIF but also serves as a model for other mega-science-class
installations looking to leverage the benefits of digital twin technology.
