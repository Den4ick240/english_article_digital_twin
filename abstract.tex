Within the sphere of MegaScience-class installations, the advent of digital
twin technology marks a paradigm shift, presenting unparalleled prospects for
advancing the design, operation, and upkeep of scientific apparatus and
systems. A digital twin, embodying a virtual counterpart of a physical entity,
facilitates real-time forecasting, optimization, surveillance, management, and
enhanced strategic decision-making in the realm of scientific equipment and
systems.

The article presents a detailed exploration into the development of digital
twins for MegaScience installations, with a special focus on the Siberian Ring
Source of Photons (SKIF). It begins by reviewing existing literature on digital
twin technology in mega-science projects, emphasizing the need for advanced
simulation and operational efficiency in these high-end scientific endeavors.
The article then delves into the application of the Digital Twin technology
specifically in the development of a digital twin for the diamond window of the
Siberian Ring Source of Photons (SKIF), a component critical for the facility's
performance. Employing a combination of COMSOL Multiphysics for accurate heat
and stress field generation for small subset of input parameters and Support
Vector Regression for real time prediction of these fields for other
parameters, the study delves into the intricacies of simulating the diamond
window's behavior under various operational scenarios. From initial Python
simulations development goes to more advanced CUDA and C++ prototypes, a shift that
significantly boosts computational efficiency. The article presents a thorough
analysis of the developmental process, highlighting the potential of digital
twins in predictive maintenance and operational optimization. Developed diamond
window Digital Twin prototype will be used in further work as a part of SKIF
front-end digital twin.
