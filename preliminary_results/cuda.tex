\subsection{Development of an Optimized Prototype Using CUDA and C++}

The quest for optimization led to the adoption of CUDA (Compute Unified Device
Architecture), a parallel computing platform and application programming
interface (API) model created by NVIDIA. CUDA allows developers to use a
CUDA-enabled graphics processing unit (GPU) for general purpose processing – an
approach known as GPGPU (General-Purpose computing on Graphics Processing
Units). By leveraging the massive parallel processing power of GPUs, CUDA makes
it possible to perform complex computations much faster than traditional
CPU-based approaches. This technology is particularly well-suited for
applications that require handling large datasets and performing high-volume
computations, which is the case with our digital twin simulation.

Building upon the insights gained from the Python prototype, a new program was
developed in C++, known for its high-performance capabilities. The key
improvement in this version is the integration of CUDA, enabling the program to
handle the intensive computational tasks more efficiently.

Similar to the Python prototype, the C++ program reads pregenerated data from
COMSOL simulations and the 3D STL model of the diamond window.

The source code from libsvm, a library for Support Vector Machines (SVM), was
adapted to run within CUDA kernels. This adaptation allows the program to
execute the NuSVR model for each point of the 3D model directly on the GPU. By
running the NuSVR model in parallel across the multiple cores of the GPU, the
program aims to dramatically increase the speed of the simulation. This
approach is expected to bring the processing time down to levels suitable for
real-time applications.

The development of the C++ and CUDA-based prototype is still underway, with
significant progress made in terms of computational efficiency. However, a
current challenge being faced is the efficient allocation of memory on the GPU
for the models. Current programm can't work with sufficient number of points
due to memory limitations, and optimizing this aspect is the next critical step
in the development process.
