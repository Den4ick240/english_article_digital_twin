\subsection{Development of a Python Prototype for Digital Twin Simulation}

In the preliminary phase of developing the digital twin for the Siberian Ring
Source of Photons (SKIF), a simple prototype has been created. This
program is designed to integrate and utilize the pre-generated data from COMSOL
Multiphysics simulations, specifically focusing on the heat and stress fields
of the diamond window and its cooling system. The core functionalities of this
program include:

\begin{itemize}

  \item Data Integration: The program reads pre-generated COMSOL data and a 3D
    STL model of the diamond window. This data includes temperature and stress
    data for each model point for different input parameters.

  \item User Interface for Parameter Input: The program features a graphical
    user interface where users can input desired parameters, such as beam
    position (x, y coordinates) and cooling water flow rate. Additionally,
    users can select which output parameter (temperature or stress) they wish
    to visualize.

  \item Machine Learning Model Training: For each point on the 3D model and for
    each output value (temperature, stress), the program trains a NuSVR
    (Nu-Support Vector Regression) model from the scikit-learn library. This
    machine learning approach allows for predictions of temperature and stress
    based on input parameters.

  \item Visualization and Prediction: Based on the inserted input parameters,
    the program predicts and displays the selected output parameter on the 3D
    model. This visualization is achieved through color mapping, providing an
    intuitive understanding of how temperature or stress varies across the
    diamond window.

  \item Model Saving and Loading: Recognizing that training machine learning
    models can be time-consuming, the program is equipped with the capability
    to save and load trained SVR models. 

\end{itemize}

The development of this program is grounded in Python, a versatile and
widely-used programming language known for its readability and extensive
support for scientific computing. Python's vast ecosystem of libraries and
tools makes it an ideal choice for developing complex simulation and data
analysis applications \cite{python2021python}.

Scikit-learn, a key library used in this program, is an open-source machine
learning library for Python. It provides simple and efficient tools for data
mining and data analysis, and it is built on NumPy, SciPy, and matplotlib.
Scikit-learn is known for its robustness and ease of use, making it a popular
choice for implementing machine learning algorithms, including support vector
machines like NuSVR which was used in the described digital twin prototype
\cite{kramer2016scikit}.

While the Python prototype successfully demonstrated the integration of COMSOL
data with machine learning for the digital twin simulation, its performance
revealed a significant limitation: the processing speed was not adequate for
real-time applications. The need for a faster, more efficient solution became
apparent, leading to the exploration of alternative technologies and
programming approaches to optimize the simulation process.
