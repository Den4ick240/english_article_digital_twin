In the evolving landscape of mega-science projects, the integration of digital
infrastructure and remote access capabilities has become increasingly crucial.
A seminal paper in this domain is \cite{Balyakin2021Digital}, which offers a
comprehensive analysis of the role and implementation of digital twins and
digital traces in mega-science facilities. This paper is instrumental in
understanding the shift towards remote accessibility in scientific research,
particularly in the context of large-scale, international scientific
collaborations.

Balyakin et al. emphasize the necessity of minimizing the physical presence of
researchers at scientific facilities, a concept exemplified by the Borexino
project's use of an independent data collection system. This approach not only
facilitates remote data acquisition but also significantly reduces the
logistical and financial burdens associated with travel to these facilities.
The paper further delves into the development of appropriate digital
infrastructure, highlighting the need for new digital elements such as data
centers and advanced processing algorithms. This infrastructure is crucial for
the seamless integration of mega-science installations into existing
e-Infrastructure, thereby enhancing the efficiency and scope of scientific
research.

Moreover, the paper discusses the legal and methodological frameworks required
to support the operation of scientific facilities in remote access mode. This
aspect is critical in addressing the challenges posed by data security,
intellectual property rights, and the nuances of international scientific
cooperation. The authors also underscore the importance of engineering
personnel in maintaining and ensuring the functionality of remote access modes,
pointing to the need for specialized skills and resources in this area.

A key contribution of this paper is its exploration of the concept of
e-Infrastructure, particularly its emergence and evolution within the European
Union. The European Open Science Cloud (EOSC) and the Go FAIR initiative are
presented as pivotal developments in this field, demonstrating the EU's
commitment to digitalizing scientific research and fostering a more
interconnected scientific community. The paper posits that while natural
sciences are the initial beneficiaries of e-Infrastructure, its most
significant impacts are likely to be seen in the field of humanities,
necessitating the development of new assessment methods and legal solutions.

In summary, "Digital Twins vs Digital Trace in Megascience Projects" by
Balyakin et al. is a foundational paper that provides valuable insights into
the digital transformation of mega-science facilities. It not only addresses
the technical and infrastructural aspects of this transformation but also
considers the broader implications for scientific research, policy, and
international collaboration. The paper's exploration of digital twins and
digital traces offers a nuanced understanding of modern data processing
techniques, making it a crucial reference for researchers and policymakers
involved in the planning and operation of large-scale scientific installations.
