\subsection{Support Vector Regression (SVR) in Simulating Diamond Window Heat
and Stress Fields}

Support Vector Regression (SVR) is a powerful machine learning method used for
regression problems. It is particularly effective in scenarios involving small
samples, nonlinearity, and high-dimensional data \cite{Guo2009Application}. SVR
has been applied in various fields, including time-series analysis,
bioprocesses, and large-scale applications \cite{Rivas-Perea2013Support}.

In the context of simulating the heat and stress fields of the diamond window
in the Siberian Ring Source of Photons (SKIF), SVR plays a crucial role. Due to
the computational intensity of direct simulations in COMSOL Multiphysics,
real-time analysis is not feasible. To overcome this limitation, SVR is
employed to train on pre-generated COMSOL data. This approach involves running
simulations in COMSOL for a range of parameters, such as different ray
positions and cooling water flow rates, to create a dataset that captures the
behavior of the diamond window under various conditions.

Once this dataset is established, SVR is used to learn the relationship between
the input parameters and the resulting heat and stress fields. The advantage of
using SVR in this scenario is its ability to make predictions much faster than
a direct COMSOL simulation. The goal is to achieve real-time prediction
capabilities, allowing for immediate assessment and adjustment of the diamond
window's conditions during SKIF operations.

The application of SVR in predicting physical processes has been demonstrated
in various studies. For instance, Lahiri and Ghanta (2008) successfully used
SVR to predict the pressure drop of slurry flow in pipelines, showcasing its
effectiveness in fluid dynamics \cite{Lahiri2008Prediction}. Similarly, Wu,
Wei, and Terpenny (2018) applied SVR to predict surface roughness in fused
deposition modeling processes, highlighting its utility in manufacturing
\cite{Wu2018Predictive}. These examples underscore the potential of SVR in
accurately predicting complex physical phenomena based on historical data.
